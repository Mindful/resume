

% ---------
% Content
% --------


\newcommand{\aboutme}{\mlt{About Me}{自己紹介}}
\newcommand{\experience}{\mlt{Experience}{職務経歴}}
\newcommand{\languages}{\mlt{Languages}{言語}}
\newcommand{\educationhistory}{\mlt{Education}{学歴}}
\newcommand{\papers}{\mlt{Research}{研究}}
\newcommand{\awards}{\mlt{Awards}{受賞歴}}
\newcommand{\extras}{\mlt{Projects}{個人開発}}




\newcommand{\header}[2]{
	\vspace*{-1cm}
	\noindent
	\iftoggle{photo}{
		\begin{tabularx}{\textwidth}{@{}lXr@{}}
	}{
		\begin{tabularx}{\textwidth}{@{}lX@{}}
	}
			\begin{minipage}{0.5\linewidth}
				\vspace*{\fill}
				{\huge{\textbf{Joshua Tanner}}} \\
				#1
			\end{minipage} &
			\hfill\begin{tabular}{l@{\hspace{5pt}}l}
				\faLinkedin & linkedin.com/in/joshuatanner2 \\
				\iftoggle{pii}{
				\faPhone & \phonenumber \\
				\faEnvelope & \email \\
				}{
				\faEnvelope & mindful.jt@gmail.com \\
				}#2
			\end{tabular} 	\iftoggle{photo}{&
			\begin{tabular}{@{}r@{}}
					\begin{minipage}{.3\textwidth}
						\begin{flushright}
							\medskip
							\fbox{\includegraphics[width=2cm, height=2cm]{photo.jpg}}
							\medskip
						\end{flushright}
					\end{minipage}
			\end{tabular}}{}
		\end{tabularx}
		
		\iftoggle{photo}{}{
		\noindent\rule{\linewidth}{1pt}\\
		}
}


\newcommand{\summarybody}{\mlt{I'm a software engineer with a focus in natural language processing and backend engineering. Programming and languages were both hobbies for me long before they became part of work, and that passion continues to fuel me professionally.}{自然言語処理とバックエンドに精通しているソフトウェアエンジニアである。プログラミングも言語も仕事に関わる前からずっと好きであり、その情熱を仕事に向けて活動している。}}

\newcommand{\lreccolingAward}{
\award{LREC-COLING 2024 Outstanding Paper Award}
}
\newcommand{\ebayexpoAward}{\award{\mlt{2016 eBay Innovation Expo Presenter}{2016 eBay社内イノベーション博覧会の発表者}}}


\newcommand{\ebaycriticaltalentAward}{\award{\mlt{2018 eBay Critical Talent award (top 1\% performance)}{2018 eBay Critical Talent award (上位1\%の社員)}}}

\newcommand{\jobexpOracle}{
	\jobexperience{\mlt{Principal Applied Scientist}{プリンシパル応用科学者}}
	{\mlt{Health Data Intelligence}{医療データ分析部}}
	{Oracle}
	{\mlt{2024/7}{2024年7月}}
	{\mlt{Current}{現在}}
	{\begin{itemize}[noitemsep]
			\item \mlt{Research into training medicine-specialized LLMs and medical entity linking}{医療に特化したLLMの学習と医療系のエンティティリンキングに関する研究}
	\end{itemize}}
}

\newcommand{\jobexpMantra}{
	\jobexperience{\mlt{NLP Lead}{NLPリード}}
	{\mlt{Development Team}{開発チーム}}
	{\mlt{Mantra Inc}{Mantra株式会社}}
	{\mlt{2021/12}{2021年12月}}
	{\mlt{2024/5}{2024年5月}}
	{\begin{itemize}[noitemsep]
			\item \mlt{Lead NLP engineering (5 engineers) across projects including:}{NLPの研究開発をリード(エンジニア5名):}
				\begin{itemize}[noitemsep]
				\item \mlt{Apply LLMs to manga translation, allowing terminology and story aware translation of >30k manga chapters}{LLMを漫画翻訳に応用し、用語やストーリーを考慮した3万話以上の翻訳を達成}
				\item \mlt{Explore training methods to improve LLM translation, pruning noisy OCR data for monolingual and bilingual fine-tuning}{ノイズの多いOCRデータを学習データから取り除き、単言語およびバイリンガルのLLMファインチューニングを実施}
				%\item \mlt{Build LLM pipeline for novel translation with LangChain to partially automate translation of 200 novels}{LangChainを用いた小説翻訳のLLMパイプラインの構築により、150冊以上の小説の翻訳を部分的に自動化}
				%\item \mlt{Integrate Mantra's translation technology into multiple multi-billion dollar companies}{Mantraの翻訳技術を複数の数十億ドル規模の企業に統合}
				\item \mlt{Train multilingual and terminology-aware translation models with Fairseq, improving BLEU on multiple languages by $>$10}{Fairseqを用いて多言語かつ用語認識可能な翻訳モデルを訓練し、複数の言語でのBLEUスコアを10点以上向上さ}
				\item \mlt{Work with professional translators to improve evaluation}{プロの翻訳者と協力し、評価方法を改善}
			\end{itemize}
			\item \mlt{Develop and maintain application backend using Django and AWS}{DjangoとAWSを使用してアプリケーションのバックエンドを開発・保守}
			%\item \mlt{Rewrite flagship mobile application from scratch in Flutter}{看板商品モバイルアプリケーションをFlutterで一から書き直し}
	\end{itemize}}
}


\newcommand{\jobexpNLPConsultant}{
	\jobexperience{\mlt{NLP Consultant}{NLPコンサルタント}}
	{\mlt{Freelance}{フリーランス}}
	{\mlt{Multiple Companies}{複数社}}
	{\mlt{2020/1}{2020年1月}}
	{\mlt{Current}{現在}}
	{\begin{itemize}[noitemsep]
		\item \mlt{Led development of an interactive translation system allowing users to translate by repeatedly selecting from model-suggested words}{翻訳モデルが提案する単語から繰り返し選択して翻訳を構築できるインタラクティブな翻訳システムの開発をリード}
		\item \mlt{Implemented optimized character-level transformer architecture in PyTorch and pretrained it with masked langauge modeling, outperforming state-of-the-art in Japanese topic classification \href{https://github.com/octanove/shiba}{[link \faExternalLinkSquare*]}}{Googleの「CANINE」論文の超効率トランスフォーマーをPyTorchで実装し、事前学習されたモデルを公開\href{https://github.com/octanove/shiba}{[link \faExternalLinkSquare*]}}
		\item \mlt{Rewrote scientific information extraction pipeline using spaCy, increasing micro f1 by more than 20 points}{spaCyを用いた情報抽出のパイプラインを書き直し、F値のマイクロ平均を20点以上改善}
	\end{itemize}}
}

\newcommand{\jobexpUTokyoIntern}{
	\jobexperience{\mlt{Research Intern}{研究実習生}}
	{\mlt{Institute of Industrial Science}{生産技術研究所}}
	{\mlt{University of Tokyo}{東京大学}}
	{\mlt{2020/12}{2020年12月}}
	{\mlt{2021/12}{2021年12月}}
	{\begin{itemize}[noitemsep]
			\item \mlt{Research into to retrieval-augmented and explainable grammatical error correction}{説明可能な検索拡張文法誤り訂正の手法の研究}
	\end{itemize}}
}


%-------------eBay---------------------


\newcommand{\ebayItemScheduler}{\item \mlt{Lead development of, tested, and maintained sole ownership of Scala distributed scheduler that powered proactive messaging for eBay's chat bot and Chinese iPhone app}{eBayのチャットボットと中国版iPhoneアプリの自発的なメッセージを送信するScalaの分散スケジューラとその自動テストの開発を率い、1人で保守}}

\newcommand{\ebayItemRevminer}{\item \mlt{Research into bootstrapping NLP process to extract informative noun/adjective pairs from product reviews was accepted for presentation at eBay's internal research conference (25\% acceptance rate)}{PythonでspaCyを利用し、商品レビューから有用な名詞と形容詞の組み合わせを抽出するブートストラップ自然言語処理を実装}}

\newcommand{\ebayItemHSR}{\item \mlt{Rearchitected the pipeline used to import shipping charge data from shipping partners, increasing the accuracy of international shipping prices on eBay's search results page by up to 40\%}{提携先の運送会社からデータをインポートするパイプラインを再構築し、eBayの検索結果ページに載る国際配送料の精度を最大で40%改善}}

\newcommand{\ebayItemAutomation}{\item \mlt{Drove an effort to automate excessive manual operations work, saving team tens of hours of work per week }{インフラやデータ整理などの手作業を自動化する取り組みを率い、国際出荷管理部チームの1週間当たりの手作業時間を数十時間減少させる事に成功}}

\newcommand{\ebayItemJava}{\item \mlt{Developed new features in Java/Spring services at scale (several million requests per hour), wrote tests using JUnit and Mockito}{JavaとSpringで大規模(毎時数百万のリクエスト )なウェブサービスにおける保守と新機能の開発をし、JUnitとMockitoを利用してその自動テストも開発}}



\newcommand{\jobexpNPD}{
	\jobexperience{\mlt{Software Engineer}{ソフトウェアエンジニア}}
	{\mlt{New Product Development \& Global Growth}{新商品開発部 \& グローバル成長部}}
	{eBay}
	{\mlt{2016/11}{2016年11月}}
	{\mlt{2019/9}{2019年9月}} 
	{\begin{itemize}[noitemsep]
		\ebayItemScheduler
		\ebayItemRevminer
		\item \mlt{Developed Ruby on Rails web service that allowed non-technical staff to write and schedule messages from chat bot to users } 
		{Ruby on Railsを利用し、非技術系社員が顧客へのメッセージをカスタマイズし送信をスケジュールできるウェブアプリケーションをゼロから開発}
	\end{itemize}}
}

\newcommand{\jobexpGSP}{
	\jobexperience{\mlt{Software Engineer}{ソフトウェアエンジニア}}
	{\mlt{Global Shipping Program}{国際配送管理部}}
	{eBay}
	{\mlt{2015/9}{2015年9月}}
	{\mlt{2016/11}{2016年11月}}
	{\begin{itemize}[noitemsep]
		\ebayItemHSR
		\ebayItemJava
		\ebayItemAutomation
	\end{itemize}}
}


\newcommand{\jobexpGDIIntern}{
	\jobexperience{\mlt{Intern}{インターン}}
	{\mlt{Global Data Infrastructure}{国際データインフラ部}}
	{eBay}
	{\mlt{2013/6}{2013年6月}}
	{\mlt{2014/9}{2014年9月}}
	{\begin{itemize}[noitemsep]
		\item \mlt{Owned and expanded functionality of Ruby on Rails application automating Hadoop permissions changes and other operational tasks}
		{Hadoopでの権限変更や運用タスクを自動化するRuby on Railsウェブアプリケーションの担当者となり、様々な新機能を開発}
	\end{itemize}}
}

% ------------ end eBay ------------------


\newcommand{\projectResolve}{
	\project{MWE Dictionary}
	{\mlt{2023}{2023年}}
	{\mlt{Multiword-expression aware dictionary for language learners using models trained with Pytorch Lightning and a gRPC backend}{
			Pytorch Lightningで訓練されたモデルとgRPCバックエンドを使用し、言語学習者向けの複単語表現を検出する辞書アプリを開発}}
}
\newcommand{\projectParasite}{
	\project{Parasite}
	{\mlt{2009}{2009年}}
	{\mlt{Lead development of StarCraft 2 mod reaching two million plays per week}{一週間当たりにプレイされた回数が2百万回に達したStarCraft 2のMODを開発し、その後維持するためにもう1人の開発者と2人のコミュニティ管理者でチームを結成}}
}

% Interpreter work

\newcommand{\interpexpSakuraconTwo}{
	\interpretingexperience{\mlt{Sakura-Con 2024}{サクラコン2024}}
	{\mlt{2024/3}{2024年3月}}
	{\mlt{Consecutive Interpretation}{逐次通訳}}
	{\mlt{Accompanied band SPYAIR, interpreting conversations with event staff }{バンドのSPYAIRに同行し、イベントスタッフとの会話を通訳}}
	
}

\newcommand{\interpexpMantra}{
\interpretingexperience
{\mlt{Mantra Inc.}{Mantra株式会社}}
{\mlt{2021 - Current}{2021年 - 現在}}
{{\mlt{Interpretation \& Translation}{逐次通訳と翻訳}}}
{\mlt{As a software engineer at a Japanese startup working on automatic Manga translation, draft technical documentation in Japanese and English, translate business communications and interpret for other staff}
{日本のベンチャー企業で漫画の自動翻訳を開発するエンジニアとして、日本語と英語で技術文書の作成、メールや広報文書の翻訳、時には社員同士の通訳を担当}
}
}

\newcommand{\interpexpCrunchyrollTwo}{
	\interpretingexperience{\mlt{Crunchyroll Expo 2022}{クランチーロールエキスポ2022}}
	{\mlt{2022/9}{2022年9月}}
	{\mlt{Consecutive Interpretation}{逐次通訳}}
	{\mlt{Accompanied band ATARASHII GAKKO!, interpreting press interviews and conversations with event staff}
		{バンドのATARASHII GAKKO!に同行し、記者会見、イベントスタッフとの会話を通訳}}
}

\newcommand{\interpexpCrunchyrollOne}{
	\interpretingexperience{\mlt{Crunchyroll Expo 2019}{クランチーロールエキスポ2019}}
	{\mlt{2019/9}{2019年9月}}
	{\mlt{Consecutive Interpretation}{逐次通訳}}
	{\mlt{Accompanied composer Yasuharu Takanashi, interpreting panel presentations, press interviews, and conversations with event staff}
	{作曲家の高梨康治に同行し、パネル発表、記者会見、イベントスタッフとの会話を通訳}}
}


\newcommand{\interpexpJonathan}{
	\interpretingexperience{\mlt{Jonathan Ng Japan Tour}{Jonathan Ng日本ツアー}}
	{\mlt{2019/8}{2019年8月}}
	{\mlt{Interpretation \& Translation}{逐次通訳と翻訳}}
	{\mlt{Accompanied jazz musician Jonathan Ng for the duration of his Japan tour, interpreting conversations between him and Japnese musicians and translating email correspondances}
	{日本ツアーの期間中、ジャズミュージシャンの\href{https://jonathanngmusic.com/}{Jonathan Ng}に同行し、日本人の演奏者との会話を通訳し、メールのやりとりも翻訳}}
}

\newcommand{\interpexpSakuraconOne}{
	\interpretingexperience{\mlt{Sakura-Con 2019}{サクラコン2019}}
	{\mlt{2019/4}{2019年4月}}
	{\mlt{Consecutive Interpretation}{逐次通訳}}
	{\mlt{Accompanied fashion designer, interpreting conversations with event staff and models}
	{ファッションデザイナーに同行し、イベントスタッフやモデルとの会話を通訳}}
}

\newcommand{\interpexpGrassroots}{
	\interpretingexperience{\mlt{28th Annual Grassroots Summit}{第28回日米草の根交流サミット大会}}
	{\mlt{2018/9}{2018年9月}}
	{\mlt{Consecutive Interpretation}{逐次通訳}}
	{\mlt{Intrepreted explanations of historic Seattle buildings into Japanese and interpreted for the tourist activities of visiting students and faculty}
	{シアトルの歴史的な建物の説明を日本語に通訳し、訪米中の教員と学生たちの観光に必要な会話も通訳}}
}

\newcommand{\translationexpMaxel}{
	\interpretingexperience{\mlt{Maxel App Localization}{Maxelアプリの日本語化}}
	{\mlt{2016/7}{2016年7月}}
	{\mlt{Translation}{翻訳}}
	{\mlt{Worked with the developer to translate the Maxel download accelerator app into Japanese}
	{開発者と協力し、ダウンロードアクセラレーターのMaxelを和訳}}
}


\newcommand{\papermweaswsd}{
	\paper{MWE as WSD: Solving Multiword Expression Identification with Word Sense Disambiguation}{\textbf{J Tanner}, J Hoffman}{EMNLP 2023 Findings \& Poster}{https://aclanthology.org/2023.findings-emnlp.14/}
}

\newcommand{\papermosla}{
	\paper{Project MOSLA: Recording Every Moment of Second Language Acquisition}{M. Hagiwara, \textbf{J Tanner}}{LREC-COLING 2024 Oral Presentation}{https://aclanthology.org/2024.lrec-main.1147/}
}

\newcommand{\papermangatranslationlongcontext}{
	\paper{Utilizing Longer Context than Speech Bubbles in Automated Manga Translation}{H Kaino, S Sugihara, T Kajiwara, T Ninomiya, \textbf{J Tanner} and S Ishiwatari}{LREC-COLING 2024 Poster}{https://aclanthology.org/2024.lrec-main.1505/}
}

\newcommand{\papergrammartagger}{
	\paper{GrammarTagger: A Multilingual, Minimally-Supervised Grammar Profiler for Language Education}{M Hagiwara, \textbf{J Tanner}, K Sakaguchi}{Preprint 2021}{https://arxiv.org/abs/2104.03190}
}

% 2019/9 - 2021/12
\newcommand{\educationUW}{
	\education{\mlt{University of Washington}{ワシントン大学}}{\mlt{ 2021}{2021年}}{\mlt{M.S. Comp. Linguistics}{計算言語学の理学修士}}
	{\mlt{Coursework in machine learning, natural language processing and linguistics}{機械学習、自然言語処理と言語学を中心とした修士課程}}
}

% 2012/9 - 2015/4
\newcommand{\educationEvergreen}{
	\education{\mlt{The Evergreen State College}{エバーグリーン州立大学}}{\mlt{2015}{2015年}}{\mlt{B.S. Computer Science}{計算機科学の理学士}}
	{\mlt{Graduated early with a major in computer science and a minor in Japanese}{コンピュータ科学の専攻と日本語の副専攻で早期卒業}}
}

% 2014年4月 - 2015年4月
\newcommand{\educationHyogo}{
	\education{\mlt{University of Hyogo}{兵庫県立大学}}{\mlt{2015}{2015年}}{\mlt{Exchange Program}{交換留学}}
	{\mlt{Studied Japanese language in a yearlong exchange program}{日本語の勉強を中心とした留学プログラム}}
}

\newcommand{\educationToudai}{ % 2020/12 - 2021/12
	\education{\mlt{University of Tokyo}{東京大学}}{\mlt{2021}{ 2021年}}{\mlt{Research Internship}{研究実習}}
	{\mlt{One year of natural language processing research at the Institute of Industrial Science}{東京大学生産技術研究所で自然言語処理の研究}}
}