

% ---------
% Content
% --------


%TODO: check the japanese resume versions
%TODO: figure out what to do with short resume
%TODO: individual TODOs below

\newcommand{\summary}{\mlt{Summary}{自己紹介}}
\newcommand{\experience}{\mlt{Experience}{職務経歴}}
\newcommand{\interpreting}{\mlt{Experience}{通訳実績}}
\newcommand{\skills}{\mlt{Technologies}{技術}}
\newcommand{\languages}{\mlt{Languages}{言語}}
\newcommand{\projects}{\mlt{Projects}{個人開発}}
\newcommand{\awards}{\mlt{Awards}{受賞歴}}
\newcommand{\projectsawards}{\mlt{Awards and Projects}{受賞歴と個人開発}}
\newcommand{\educationhistory}{\mlt{Education}{学歴}}
\newcommand{\courses}{\mlt{Courses}{授業}}
\newcommand{\graduatecoursework}{\mlt{Graduate Coursework}{大学院の課題}}
\newcommand{\skillslanguage}{\mlt{Languages}{言語}}

%TODO: fix the japanese here
\newcommand{\skillsproficient}{\mlt{Proficient}{堪能}}
\newcommand{\skillsfamiliar}{\mlt{Familiar}{arg2}}

\newcommand{\proficientskills}{
Python &  spaCy \\ 
Scala & Java \\
Ruby &   \\ 
}

\newcommand{\familiarskills}{
AllenNLP & fairseq\\
Pytorch & Play Framework\\
SQL & Ruby on Rails \\
}

\newcommand{\header}[2]{
	\vspace*{-1cm}
	\noindent
	\iftoggle{photo}{
		\begin{tabularx}{\textwidth}{@{}lXr@{}}
	}{
		\begin{tabularx}{\textwidth}{@{}lX@{}}
	}
			\begin{tabular}{@{}l@{}}
				%	{\huge{\textbf{\mlt{Joshua Tanner}{\ruby[g]{Joshua}{ジョシュア} \ruby[g]{Tanner}{ターナー}}}}} \\
				{\huge{\textbf{Joshua Tanner}}} \\
				#1
			\end{tabular} &
			\hfill\begin{tabular}{l@{\hspace{5pt}}l}
				\faPhone & \phonenumber \\
				\faEnvelope & \email \\
				\faLinkedin & \linkedin \\
				#2
			\end{tabular} 	\iftoggle{photo}{&
			\begin{tabular}{@{}r@{}}
					\begin{minipage}{.3\textwidth}
						\begin{flushright}
							\medskip
							\iftoggle{long}{
								\fbox{\includegraphics[width=2.75cm, height=2.75cm]{photo.jpg}}
							}{
								\fbox{\includegraphics[width=2cm, height=2cm]{photo.jpg}}
							}
							\medskip
						\end{flushright}
					\end{minipage}
			\end{tabular}}{}
		\end{tabularx}
		
		\iftoggle{photo}{}{
		\noindent\rule{\linewidth}{1pt}\\
		}
}


\newcommand{\coursemachinelearning}{def}
\newcommand{\sidebarcoursework}{
			%	\textbf{\mlt{(by June 2020)}{(2020年6月までに修了)}:\\}
				\sidesubsection{\mlt{Graduate}{大学院}}\\[0.25\baselineskip]
				\begin{tabular}{@{}p{\linewidth}@{}}
				\mlt{NLP 1 (Shallow Processing)}{自然言語処理 I(表層処理)}   \\
				\mlt{NLP 2 (Deep Processing)}{自然言語処理 II (深層処理)}  \\
				\mlt{Machine Learning for NLP}{自然言語処理向けの機械学習}  \\
				\mlt{NLP Systems and Applications}{NLPシステムの開発と応用} \\
				\mlt{Neural Language Models}{ニューラル言語モデルの解析} \\
				\mlt{Information Extraction}{情報抽出} \\
				\mlt{English Syntactic Theory}{英語の統語論} \\
				\mlt{Japanese Syntax and Semantics}{日本語の統語論と意味論}\\
				\mlt{Phonetics}{音声学}
				\end{tabular} \\
			\sidesubsection{\mlt{Undergraduate}{大学}}
				\begin{tabular}{@{}p{\linewidth}@{}}
				\mlt{Operating Systems}{オペレーティングシステム}   \\
				\mlt{Networks}{ネットワーク} \\
				\mlt{Android Development}{アンドロイド開発} \\
				\mlt{Japanse 1 \& 2}{日本語 I &\& II}
				\end{tabular} \\[0.25\baselineskip]
}

%TODO - no japanese on these three
\newcommand{\courseworkNLP}{
	\project{NLP 1 \& 2}{
	\mlt{Implemented NLP algorithms such as n-gram language models, HMM part of speech tagging, probabilistic context free grammar generation and CKY parsing.}
	{工事中}}
}
\newcommand{\courseworkML}{
	\project{Machine Learning for NLP}{
	\mlt{Implemented machine learning algorithms for text classification such as decision trees, naive bayes, and KNN. Partial implementations and other work on maximum entropy models, support vector machines and neural networks.}
	{工事中}}
}
\newcommand{\courseworkNLM}{
	\project{Neural Language Models}{
	\mlt{Designed an experiment to determine how well BERT and other neural language models comprehend idiom paraphrases, fine tuning the models and training probing classifiers with PyTorch. }
	{工事中}}
}


%TODO: rewrite this for NLP
\newcommand{\summarybody}{\mlt{I’m a software engineer with a focus on backend work and complex systems. Programming has been a hobby for me since long before it became my job, but my passion for quality of work has remained constant. I also have a soft spot for linguistics and enjoy learning languages.}{バックエンド作業と複雑なシステムに取り組む事を楽しめる開発者である。仕事でプログラミングをする前からずっと趣味でコードを書き、コードの質にこだわる事を最初から続けている。言語学と語学にも興味を持つ。}}



\newcommand{\jobexpOctanoveIntern}{
	\jobexperienceNoTeam{\mlt{Research Intern}{研究インターン}}
	{Octanove Labs}
	{\mlt{June 2020}{2020年6月}}
	{\mlt{October 2020}{2020年10月}}
	{\begin{itemize}[noitemsep]
		\item \mlt{Train multilingual multitask seq2seq models for translation and grammatical error correction using fairseq}{fairseqで文法誤り訂正と自動翻訳をする多言語マルチタスクseq2seqモデルを学習}
		\item \mlt{Implement an ensemble of language identification models to clean and filter data with heavy code switching}{コードスイッチングの多いデータをフィルターしてきれいにするために言語識別モデルのアンサンブルを実装}
	\end{itemize}}
}

%-------------eBay---------------------


\newcommand{\ebayItemScheduler}{\item \mlt{Lead development of, tested, and maintained sole ownership of the Scala distributed scheduler that powered proactive messaging for eBay ShopBot and eBay's Chinese iPhone app}{eBayのチャットボットと中国版iPhoneアプリの自発的なメッセージを送信するScalaの分散スケジューラとその自動テストの開発を率い、一人で保守}}

\newcommand{\ebayItemRevminer}{\item \mlt{Implemented a bootstrapping NLP process to extract informative noun/adjective pairs from reviews, using Python and spaCy (based on “RevMiner" paper)}{PythonでspaCyを利用し、レビューから有用な名詞と形容詞の組み合わせを抽出するブートストラップ自然言語処理を実装 (“RevMiner”という論文に基づき)}}

\newcommand{\ebayItemHSR}{\item \mlt{Rearchitected the pipeline used to import shipping charge data from shipping partners, increasing the accuracy of shipping prices on eBay's search results page by up to 40\%}{提携先の運送会社からデータをインポートするパイプラインを再構築し、eBayの検索結果ページに載る配送料の精度を最大で40%改善}}

\newcommand{\ebayItemAutomation}{\item \mlt{Drove an effort to automate excessive manual operations work, saving the Global Shipping Program team tens of hours of work per week }{インフラやデータ整理などの手作業を自動化する取り組みを率い、国際出荷管理部チームの一週間当たりの手作業時間を数十時間減少させる事に成功}}

\newcommand{\ebayItemJava}{\item \mlt{Maintained and developed new features in Java/Spring services at scale (several million requests per hour), wrote tests using JUnit and Mockito}{JavaとSpringで大規模(毎時数百万のリクエスト )のウェブサービスにおける保守と新機能の開発をし、JUnitとMockitoを利用しその自動てテストも開発}}



%TODO: here and below in the GSP specific item, 国際出荷管理 is almost certainly wrong. maybe 国際郵送開発?
\newcommand{\jobexpEbay}{
	\jobexperience{\mlt{Software Engineer}{システムエンジニア}}
	{\mlt{Global Shipping Program}{国際出荷管理部} \& \mlt{New Product Development}{新商品開発部}}
	{eBay}
	{\mlt{September 2015}{2015年9月}}
	{\mlt{September 2019}{2019年9月}}
	{\begin{itemize}[noitemsep]
		\ebayItemScheduler
		\ebayItemRevminer
		\ebayItemHSR
		\ebayItemAutomation
		\ebayItemJava
	\end{itemize}}
}



\newcommand{\jobexpNPD}{
	\jobexperience{\mlt{Software Engineer}{システムエンジニア}}
	{\mlt{New Product Development \& Global Growth}{新商品開発部 \& グローバル成長部}}
	{eBay}
	{\mlt{November 2016}{2016年11月}}
	{\mlt{September 2019}{2019年9月}} 
	{\begin{itemize}[noitemsep]
		\ebayItemScheduler
		\ebayItemRevminer
		\item \mlt{Developed a Ruby on Rails web service that allowed non-technical users to customize and schedule proactive messages from ShopBot to users }
		{Ruby on Railsを利用し、非技術系社員が顧客へのメッセージをカスタマイズしたりスケジュール送信できるウェブアプリケーションをゼロから開発}
	\end{itemize}}
}

\newcommand{\jobexpGSP}{
	\jobexperience{\mlt{Software Engineer}{システムエンジニア}}
	{\mlt{Global Shipping Program}{国際出荷管理部}}
	{eBay}
	{\mlt{September 2015}{2015年9月}}
	{\mlt{November 2016}{2016年11月}}
	{\begin{itemize}[noitemsep]
		\ebayItemHSR
		\ebayItemJava
		\ebayItemAutomation
	\end{itemize}}
}


\newcommand{\jobexpGDIIntern}{
	\jobexperience{\mlt{Intern}{インターン}}
	{\mlt{Global Data Infrastructure}{国際データインフラ部}}
	{eBay}
	{\mlt{June 2013}{2013年6月}}
	{\mlt{September 2014}{2014年9月}}
	{\begin{itemize}[noitemsep]
		\item \mlt{Took ownership of and greatly expanded the functionality of a Ruby on Rails application automating Hadoop permissions changes and other operational tasks}
		{Hadoopでの権限変更や運用タスクを自動化するRuby on Railsウェブアプリケーションの担当者となり、様々な新機能を開発}
	\end{itemize}}
}

% ------------ end eBay ------------------

\newcommand{\projectParasite}{
	\project{Parasite}{
	\mlt{A mod for StarCraft 2 which at peak averaged 2 million plays per week.  Written in a combination of a native GUI language and Galaxy, a C-like language. I was responsible for the architecture, design, and vast majority of the code written, eventually onboarding another developer and two community managers.}
	{一週間当たりにプレイされた回数が2百万回に達したStarCraft 2のMODである。専用のGUIプログラミング言語とGalaxyというC言語系プログラミング言語で開発を行った。一人で設計とデザインを考え出し、コードの大部分も一人で書いた。後でもう一人の開発者と二人のコミュニティ管理者でチームを結成した。}}
}

\newcommand{\projectPytext}{
	\project{Pytext}{
	\mlt{A desktop application written in Python which wraps communication with an email server and use of email to SMS gateways to allow users to participate in SMS conversations without the need for a phone number. }
	{ユーザーの入力したメッセージをメールとしてSMSゲートウェイに送信する事により、電話番号なしでSMS会話に参加する事を可能にするデスクトップアプリケーションである。一人でPythonで開発した。}}
}



%TODO - languages out of sync
\newcommand{\awardParasite}{
	\award{Parasite}
	{\mlt{Personal Project}{個人開発}}
	{\mlt{2009}{2009年}}
	{\mlt{Developed a StarCraft 2 mod which at peak averaged two million plays per week, later onboarding another developer and two community managers.}{一週間当たりにプレイされた回数が2百万回に達したStarCraft 2のMODである。専用のGUIプログラミング言語とGalaxyというC言語系プログラミング言語で開発を行った。一人で設計とデザインを考え出し、コードの大部分も一人で書いた。後でもう一人の開発者と二人のコミュニティ管理者でチームを結成した。}}
}

%TODO: languages may be slightly out of sync here
\newcommand{\awardAmplify}{
	\award{\mlt{eBay Amplify Presenter}{eBay Amplifyの登壇者に選出}}
	{\mlt{Paper Submission}{論文投稿}}
	{\mlt{2018/05/7}{2018年5月7日}}
	{\mlt{A paper based on information extraction research I had done as a side project was accepted at eBay's internal research conference with 25\% acceptance rate.}{片手間で進めた情報抽出の研究に基づいた論文でeBayの社内学会の登壇者に選ばれた (25%のアクセプト率)。}}
}


%TODO: languages out of sync here
\newcommand{\awardGoldenTicket}{
		\award{\mlt{Innovation Expo Presenter}{イノベーション博覧会の発表者に選出}}
		{eBay Hack Week}
		{\mlt{2016/06/24}{2016年06月24日}}
		{\mlt{Our service which selected n-grams from web pages to turn into affiliate links was one of two chosen to be presented from the 14 candidate projects.}
		{ウェブページのnグラムをページ内容への関連性に基づくヒューリスティックで選択し、アフィリエイトリンクに変換するNode.jsサービスを開発し、Hack Weekに参加した14チーム中から博覧会で作品を発表する2チームの1つに選ばれた。}}
}

\newcommand{\awardEbayApi}{
		\award{\mlt{Best eBay API Integration}{eBay API利用部門で優勝}}
		{2013 CodeDay Seattle}
		{\mlt{2013/05/26}{2013年05月26日}}
		{\mlt{Placed first in category among 7 teams by building a Python/Flask web application which signed into eBay and purchased random items from eBay on behalf of the user. }
		{ユーザーとしてeBayにログインし自動的にランダムな商品を購入するウェブアプリケーションをPythonで開発し、部門で7つのチームの中から優勝した。}}
}



% Interpreter work

\newcommand{\interpexpSakuraconTwo}{
	\interpretingexperience{\mlt{Sakura Con 2020}{桜コン2020}}
	{\mlt{April 2020}{2020年4月}}
	{\mlt{blah blah blah}
	{ブラブラブラ}
	}

}

\newcommand{\interpexpCrunchyrollOne}{
	\interpretingexperience{\mlt{Crunchyroll Expo 2019}{クランチーロールエキスポ2019}}
	{\mlt{September 2019}{2019年9月}}
	{\mlt{Consecutive Interpretation}{逐次通訳}}
	{\mlt{Accompanied visiting composer Yasuharu Takanashi for the duration of the conference, interpreting panel presentations, press interviews, and convresations with even staff.}
	{大会の期間中、来客された作曲の高梨康治に同行し、パネル発表、記者会見、イベントスタフとの会話を通訳}}
}

\newcommand{\interpexpJonathan}{
	\interpretingexperience{\mlt{Jonathan Ng Japan Tour}{Jonathan Ng日本ツアー}}
	{\mlt{August 2019}{2019年8月}}
	{\mlt{Consecutive Interpretation, Translation}{逐次通訳、翻訳}}
	{\mlt{Accompanied jazz musician Jonathan Ng for the duration of his Japan tour, interpreting conversations between him and Japnese musicians and translating email correspondances.}
	{日本ツアーの期間中、ジャズミュージシャンのJonathan Ngに同行し、日本人の演奏者との会話を通訳し、メールのやりとりも翻訳}}
}

\newcommand{\interpexpSakuraconOne}{
	\interpretingexperience{\mlt{Sakura Con 2019}{桜コン2019}}
	{\mlt{April 2019}{2019年4月}}
	{\mlt{Consecutive Interpretation}{逐次通訳}}
	{\mlt{Accompanied visiting fashion designer for the duration of the conference, interpreting conversations with event staff and models.}
	{大会の期間中、来客されたファションデザイナーに同行し、イベントスタフやモデルとの会話を通訳}}
}

\newcommand{\interpexpGrassroots}{
	\interpretingexperience{\mlt{28th Annual Grassroots Summit}{第28回日米草の根交流サミット大会}}
	{\mlt{September 2018}{2018年9月}}
	{\mlt{Consecutive Interpretation}{逐次通訳}}
	{\mlt{Intrepreted explanations of historic Seattle buildings into Japanese and interpreted for the tourist activities of visiting students and faculty.}
	{シアトルの歴史的な建物の説明を日本語に通訳し、訪米中の教員と学生たちの観光に必要な会話も通訳}}
}




\newcommand{\educationUW}{
	\education{\mlt{University of Washington}{ワシントン大学}}{\mlt{2019/9 – 2021/12}{2019年9月 – 2021年12月}}{\mlt{M.S. Computational Linguistics}{計算言語学の理学修士}}
	{\mlt{(Expected) masters degree in computational linguistics.}{(見込み)計算言語学の修士号}}
}
\newcommand{\educationEvergreen}{
	\education{\mlt{The Evergreen State College}{エバーグリーン州立大学}}{\mlt{2012/9 – 2015/4}{2012年9月 – 2015年4月}}{\mlt{B.S. Computer Science}{コンピュータ科学の理学士}}
	{\mlt{Graduated early with a major in computer science and a minor in Japanese.}{コンピュータ科学の専攻と日本語の副専攻で早期卒業}}
}

\newcommand{\educationHyogo}{
	\education{\mlt{University of Hyogo}{兵庫県立大学}}{\mlt{2014/4 – 2015/4}{2014年4月 – 2015年4月}}{\mlt{Exchange Program}{交換留学}}
	{\mlt{Studied Japanese language in a yearlong exchange program.}{日本語の勉強を中心とした留学プログラム}}
}

\newcommand{\educationToudai}{def}

\newcommand{\skillenglish}{\mlt{English - Native}{英語 - 母国語}}
\newcommand{\skilljapanese}{\mlt{Japanese - Fluent (JLPT N1)}{日本語 - 日本語能力試験N1}}