

% ---------
% Content
% --------


\newcommand{\aboutme}{\mlt{About Me}{自己紹介}}
\newcommand{\experience}{\mlt{Experience}{職務経歴}}
\newcommand{\interpreting}{\mlt{Experience}{通訳実績}}
\newcommand{\languages}{\mlt{Languages}{言語}}
\newcommand{\educationhistory}{\mlt{Education}{学歴}}
\newcommand{\papers}{\mlt{Academic Papers}{学術論文}}
\newcommand{\extras}{\mlt{Awards \& Projects}{学業}}




\newcommand{\header}[2]{
	\vspace*{-1cm}
	\noindent
	\iftoggle{photo}{
		\begin{tabularx}{\textwidth}{@{}lXr@{}}
	}{
		\begin{tabularx}{\textwidth}{@{}lX@{}}
	}
			\begin{minipage}{0.5\linewidth}
				\vspace*{\fill}
				{\huge{\textbf{Joshua Tanner}}} \\
				#1
			\end{minipage} &
			\hfill\begin{tabular}{l@{\hspace{5pt}}l}
				\iftoggle{pii}{\faPhone & \phonenumber \\
				\faEnvelope & \email \\
				\faLinkedin & \linkedin \\
				#2}{}
			\end{tabular} 	\iftoggle{photo}{&
			\begin{tabular}{@{}r@{}}
					\begin{minipage}{.3\textwidth}
						\begin{flushright}
							\medskip
							\fbox{\includegraphics[width=2cm, height=2cm]{photo.jpg}}
							\medskip
						\end{flushright}
					\end{minipage}
			\end{tabular}}{}
		\end{tabularx}
		
		\iftoggle{photo}{}{
		\noindent\rule{\linewidth}{1pt}\\
		}
}


\newcommand{\summarybody}{\mlt{I'm a software engineer with a focus in natural language processing and backend engineering. Programming and languages were both hobbies for me long before they became part of work, and that passion continues to fuel me professionally.}{自然言語処理とバックエンドに精通しているソフトウェアエンジニアである。プログラミングも言語も仕事に関わる前からずっと好きであり、その情熱を仕事に向けて活動している。}}

\newcommand{\jobexpMantra}{
	\jobexperience{\mlt{NLP Engineer}{NLP Engineer}}
	{\mlt{Development Team}{開発チーム}}
	{\mlt{Mantra Inc}{Mantra株式会社}}
	{\mlt{2021/12}{2021年12月}}
	{\mlt{Current}{現在}}
	{\begin{itemize}[noitemsep]
			\item \mlt{Mentor engineers working on manga translation system through projects using Pytorch and LangChain such as applying and fine-tuning large language models, glossary-aware translation and fluency evaluation to achieve TODO}{arg2}
			\item \mlt{Defined and implemented company policy for information security and engineer hiring}{arg2}
			\item \mlt{Rewrote flagship mobile application from scratch in Flutter }{arg2}
			\item \mlt{Develop and maintain application backend using Django and AWS}{arg2}
	\end{itemize}}
}


\newcommand{\jobexpNLPConsultant}{
	\jobexperience{\mlt{NLP Consultant}{NLPコンサルタント}}
	{\mlt{Freelance}{フリーランス}}
	{\mlt{Multiple Companies}{数社}}
	{\mlt{2020/1}{2020年1月}}
	{\mlt{Current}{現在}}
	{\begin{itemize}[noitemsep]
		\item \mlt{Lead development of interactive translation system allowing users to choose from output words at each decoding step of Huggingface transformer models}{arg2}
		\item \mlt{Implemented optimized character-level transformer architecture in PyTorch and pretrained it with masked langauge modeling, achieving state-of-the-art in Japanese sentiment classification}{ググルの「CANINE」論文の超効率トランスフォーマーをPyTorchで実装し、\href{https://github.com/octanove/shiba}{事前学習されたモデルを公開}}
		\item \mlt{Rewrote scientific information extraction pipeline using spaCy, increasing micro f1 by more than 20 points}{spaCyを用いた情報抽出のパイプラインを書き直し、F値のマイクロ平均を20点以上改善}
	\end{itemize}}
}

\newcommand{\jobexpUTokyoIntern}{
	\jobexperience{\mlt{Research Intern}{研究実習生}}
	{\mlt{Institute of Industrial Science}{生産技術研究所}}
	{\mlt{University of Tokyo}{東京大学}}
	{\mlt{2020/12}{2020年12月}}
	{\mlt{2021/12}{2021年12月}}
	{\begin{itemize}[noitemsep]
			\item \mlt{Research into deep learning approaches to retrieval-augmented and explainable grammatical error correction}{arg2}
	\end{itemize}}
}


%-------------eBay---------------------


\newcommand{\ebayItemScheduler}{\item \mlt{Lead development of, tested, and maintained sole ownership of the Scala distributed scheduler that powered proactive messaging for eBay ShopBot and eBay's Chinese iPhone app}{eBayのチャットボットと中国版iPhoneアプリの自発的なメッセージを送信するScalaの分散スケジューラとその自動テストの開発を率い、一人で保守}}

\newcommand{\ebayItemRevminer}{\item \mlt{Research implementing a bootstrapping NLP process to extract informative noun/adjective pairs from product reviews was accepted for presentation at eBay's internal research conference (25\% acceptance rate)}{PythonでspaCyを利用し、商品レビューから有用な名詞と形容詞の組み合わせを抽出するブートストラップ自然言語処理を実装}}

\newcommand{\ebayItemHSR}{\item \mlt{Rearchitected the pipeline used to import shipping charge data from shipping partners, increasing the accuracy of international shipping prices on eBay's search results page by up to 40\%}{提携先の運送会社からデータをインポートするパイプラインを再構築し、eBayの検索結果ページに載る国際配送料の精度を最大で40%改善}}

\newcommand{\ebayItemAutomation}{\item \mlt{Drove an effort to automate excessive manual operations work, saving the Global Shipping Program team tens of hours of work per week }{インフラやデータ整理などの手作業を自動化する取り組みを率い、国際出荷管理部チームの一週間当たりの手作業時間を数十時間減少させる事に成功}}

\newcommand{\ebayItemJava}{\item \mlt{Maintained and developed new features in Java/Spring services at scale (several million requests per hour), wrote tests using JUnit and Mockito}{JavaとSpringで大規模(毎時数百万のリクエスト )なウェブサービスにおける保守と新機能の開発をし、JUnitとMockitoを利用してその自動てテストも開発}}



\newcommand{\jobexpNPD}{
	\jobexperience{\mlt{Software Engineer}{ソフトウェアエンジニア}}
	{\mlt{New Product Development \& Global Growth}{新商品開発部 \& グローバル成長部}}
	{\href{https://www.ebay.com/}{eBay}}
	{\mlt{2016/11}{2016年11月}}
	{\mlt{2019/9}{2019年9月}} 
	{\begin{itemize}[noitemsep]
		\ebayItemScheduler
		\ebayItemRevminer
		\item \mlt{Developed a Ruby on Rails web service that allowed non-technical users to customize and schedule proactive messages from ShopBot to users }
		{Ruby on Railsを利用し、非技術系社員が顧客へのメッセージをカスタマイズしたりスケジュール送信できるウェブアプリケーションをゼロから開発}
	\end{itemize}}
}

\newcommand{\jobexpGSP}{
	\jobexperience{\mlt{Software Engineer}{ソフトウェアエンジニア}}
	{\mlt{Global Shipping Program}{国際配送管理部}}
	{\href{https://www.ebay.com/}{eBay}}
	{\mlt{2015/9}{2015年9月}}
	{\mlt{2016/11}{2016年11月}}
	{\begin{itemize}[noitemsep]
		\ebayItemHSR
		\ebayItemJava
		\ebayItemAutomation
	\end{itemize}}
}


\newcommand{\jobexpGDIIntern}{
	\jobexperience{\mlt{Intern}{インターン}}
	{\mlt{Global Data Infrastructure}{国際データインフラ部}}
	{\href{https://www.ebay.com/}{eBay}}
	{\mlt{2013/6}{2013年6月}}
	{\mlt{2014/9}{2014年9月}}
	{\begin{itemize}[noitemsep]
		\item \mlt{Owned and greatly expanded the functionality of a Ruby on Rails application automating Hadoop permissions changes and other operational tasks}
		{Hadoopでの権限変更や運用タスクを自動化するRuby on Railsウェブアプリケーションの担当者となり、様々な新機能を開発}
	\end{itemize}}
}

% ------------ end eBay ------------------


\newcommand{\extraParasite}{
	\extra{Parasite}
	{\mlt{Personal Project / 2009}{2009年}}
	{\mlt{Lead development of StarCraft 2 mod reaching two million plays per week.}{一週間当たりにプレイされた回数が2百万回に達したStarCraft 2のMODを開発し、その後維持するためにもう一人の開発者と二人のコミュニティ管理者でチームを結成した。}}
}

%\newcommand{\awardAmplify}{
	%\extra{\mlt{eBay Amplify Presenter}{eBay Amplifyの登壇者に選出}}
	%{\mlt{Paper Submission}{論文投稿}}
	%{\mlt{2018/05/7}{2018年5月7日}}
	%{\mlt{Information extraction research I conducted as a side project was accepted for presentation at eBay's internal research conference (25\% acceptance rate).}{片手間で進めた情報抽出の研究でeBayの社内学会の登壇者に選ばれた (25\%のアクセプト率)。}}
%}


\newcommand{\extraInnovationExpo}{
		\extra{\mlt{Expo Presenter}{社内博覧会の発表者に選出}}
		{\mlt{eBay Innovation Expo / 2016}{2016年}} %2016/06/04
		{\mlt{Hackathon project converting relevant n-grams into affiliate links featured in internal expo (14\% acceptance rate).}
		{ハッカソンのチームで開発した、ページ内容と関連性の高いnグラムをアフィリエイトリンクに変換するNode.jsサービスが社内イノベーション博覧会での発表に選ばれた(14\%のアクセプト率)。}}
}

\newcommand{\awardEbayApi}{
		\extra{\mlt{Best eBay API Integration}{eBay API利用部門で優勝}}
		{2013 CodeDay Seattle}
		{\mlt{2013/05/26}{2013年05月26日}}
		{\mlt{Placed first in category among 7 teams by building a Python/Flask web application which signed into eBay and purchased random items from eBay on behalf of the user. }
		{ユーザーとしてeBayにログインし自動的にランダムな商品を購入するウェブアプリケーションをPythonで開発し、部門で7つのチームの中から優勝した。}}
}



% Interpreter work

\newcommand{\interpexpMantra}{
\interpretingexperience
{\mlt{Mantra Inc.}{Mantra株式会社}}
{\mlt{2021 - Current}{2021年 - 現在}}
{{\mlt{Interpretation \& Translation}{逐次通訳と翻訳}}}
{\mlt{As a software engineer at a Japanese startup, my work included drafting technical documentation in both Japanese and English, translating business communications and occasionally interpreting for other staff}
{日本のベンチャー企業で漫画の自動翻訳を開発するエンジニアとして、日本語と英語で技術文書の作成、メールや広報文書の翻訳、時には社員同士の通訳を担当}
}
}

\newcommand{\interpexpCrunchyrollTwo}{
	\interpretingexperience{\mlt{Crunchyroll Expo 2022}{クランチーロールエキスポ2022}}
	{\mlt{2022/9}{2022年9月}}
	{\mlt{Consecutive Interpretation}{逐次通訳}}
	{\mlt{Accompanied band ATARASHII GAKKO!, interpreting press interviews and conversations with event staff}
		{バンドのATARASHII GAKKO!に同行し、記者会見、イベントスタッフとの会話を通訳}}

}

\newcommand{\interpexpCrunchyrollOne}{
	\interpretingexperience{\mlt{Crunchyroll Expo 2019}{クランチーロールエキスポ2019}}
	{\mlt{2019/9}{2019年9月}}
	{\mlt{Consecutive Interpretation}{逐次通訳}}
	{\mlt{Accompanied composer Yasuharu Takanashi, interpreting panel presentations, press interviews, and conversations with event staff}
	{作曲家の高梨康治に同行し、パネル発表、記者会見、イベントスタッフとの会話を通訳}}
}


\newcommand{\interpexpJonathan}{
	\interpretingexperience{\mlt{Jonathan Ng Japan Tour}{Jonathan Ng日本ツアー}}
	{\mlt{2019/8}{2019年8月}}
	{\mlt{Interpretation \& Translation}{逐次通訳と翻訳}}
	{\mlt{Accompanied jazz musician \href{https://jonathanngmusic.com/}{Jonathan Ng} for the duration of his Japan tour, interpreting conversations between him and Japnese musicians and translating email correspondances}
	{日本ツアーの期間中、ジャズミュージシャンの\href{https://jonathanngmusic.com/}{Jonathan Ng}に同行し、日本人の演奏者との会話を通訳し、メールのやりとりも翻訳}}
}

\newcommand{\interpexpSakuraconOne}{
	\interpretingexperience{\mlt{Sakura Con 2019}{桜コン2019}}
	{\mlt{2019/4}{2019年4月}}
	{\mlt{Consecutive Interpretation}{逐次通訳}}
	{\mlt{Accompanied fashion designer, interpreting conversations with event staff and models}
	{ファッションデザイナーに同行し、イベントスタッフやモデルとの会話を通訳}}
}

\newcommand{\interpexpGrassroots}{
	\interpretingexperience{\mlt{28th Annual Grassroots Summit}{第28回日米草の根交流サミット大会}}
	{\mlt{2018/9}{2018年9月}}
	{\mlt{Consecutive Interpretation}{逐次通訳}}
	{\mlt{Intrepreted explanations of historic Seattle buildings into Japanese and interpreted for the tourist activities of visiting students and faculty}
	{シアトルの歴史的な建物の説明を日本語に通訳し、訪米中の教員と学生たちの観光に必要な会話も通訳}}
}

\newcommand{\translationexpMaxel}{
	\interpretingexperience{\mlt{Maxel App Localization}{Maxelアプリの日本語化}}
	{\mlt{2016/7}{2016年7月}}
	{\mlt{Translation}{翻訳}}
	{\mlt{Worked with the developer to translate the \href{https://maxelapp.com/}{Maxel} download accelerator app into Japanese}
	{開発者と協力し、ダウンロードアクセラレーターの\href{https://maxelapp.com/}{Maxel}を和訳}}
}


\newcommand{\papermweaswsd}{
	\paper{MWE as WSD: Solving Multiword Expression Identification with Word Sense Disambiguation}{\textbf{J Tanner}, J Hoffman}{[in review]}{https://openreview.net/pdf?id=qWlU7Y8BF4S}
}

\newcommand{\papergrammartagger}{
	\paper{GrammarTagger: A Multilingual, Minimally-Supervised Grammar Profiler for Language Education}{M Hagiwara, \textbf{J Tanner}, K Sakaguchi}{arxiv preprint, 2021}{https://arxiv.org/abs/2104.03190}
}

% 2019/9 - 2021/12
\newcommand{\educationUW}{
	\education{\mlt{University of Washington}{ワシントン大学}}{\mlt{ 2021}{2021年}}{\mlt{M.S. Comp. Linguistics}{計算言語学の理学修士}}
	{\mlt{Coursework in machine learning and natural language processing}{機械学習と自然言語処理を中心とした修士課程}}
}

% 2012/9 - 2015/4
\newcommand{\educationEvergreen}{
	\education{\mlt{The Evergreen State College}{エバーグリーン州立大学}}{\mlt{2015}{2015年}}{\mlt{B.S. Computer Science}{計算機科学の理学士}}
	{\mlt{Graduated early with a major in computer science and a minor in Japanese}{コンピュータ科学の専攻と日本語の副専攻で早期卒業}}
}

% 2014年4月 - 2015年4月
\newcommand{\educationHyogo}{
	\education{\mlt{University of Hyogo}{兵庫県立大学}}{\mlt{2015}{2015年}}{\mlt{Exchange Program}{交換留学}}
	{\mlt{Studied Japanese language in a yearlong exchange program}{日本語の勉強を中心とした留学プログラム}}
}

\newcommand{\educationToudai}{ % 2020/12 - 2021/12
	\education{\mlt{University of Tokyo}{東京大学}}{\mlt{2021}{ 2021年}}{\mlt{Research Internship}{研究実習}}
	{\mlt{One year of natural language processing research at the Institute of Industrial Science}{東京大学生産技術研究所で自然言語処理の研究実習}}
}