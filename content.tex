

% ---------
% Content
% --------

\newcommand{\summary}{\mlt{Summary}{自己紹介}}
\newcommand{\experience}{\mlt{Experience}{職務経歴}}
\newcommand{\skills}{\mlt{Skills}{技術}}
\newcommand{\languages}{\mlt{Languages}{言語}}
\newcommand{\projects}{\mlt{Projects}{個人開発}}
\newcommand{\awards}{\mlt{Awards}{受賞歴}}
\newcommand{\projectsawards}{\mlt{Projects and Awards}{受賞歴と個人開発}}
\newcommand{\educationhistory}{\mlt{Education}{学歴}}

\newcommand{\skillsprogramminglanguages}{\mlt{Programming Languages}{プログラミング言語}}
\newcommand{\skillsframeworks}{\mlt{Frameworks}{フレームワーク}}
\newcommand{\skillsmisc}{\mlt{Other Technology}{その他の専門技術}}
\newcommand{\skillslanguage}{\mlt{Natural Languages}{自然言語}}


\newcommand{\coursework}{
			%Temporary US only coursework section
			\iftoggle{en}{
				\majorsection{Relevant Coursework} \halflinebreak\\
				\begin{tabular}{@{}l l@{}}
				\multicolumn{2}{c}{\large{Fall 2019}}\\
				LING 570 & Shallow Processing Techniques For Natural Language Processing  \\
				LING 566 & Introduction To Syntax For Computational Linguistics\vspace{0.1cm} \\
				\multicolumn{2}{c}{\large{Winter 2020}} \\
				LING 571 & Deep Processing Techniques for Natural Language Processing  \\
				LING 572 & Advanced Statistical Methods in Natural Language Processing  \\
				LING 575 & Analyzing Neural Language Models\vspace{0.1cm} \\
				\multicolumn{2}{c}{\large{Spring 2020}} \\
				LING 573 & Natural Language Processing Systems and Applications \\
				LING 575 & Clinical Natural Language Processing \\
				\end{tabular}
			}{}
}

\newcommand{\summarybody}{\mlt{I’m a software engineer with a focus on backend work and complex systems. Programming has been a hobby for me since long before it became my job, but my passion for quality of work has remained constant. I also have a soft spot for linguistics and enjoy learning languages.}{バックエンド作業と複雑なシステムに取り組む事を楽しめる開発者である。仕事でプログラミングをする前からずっと趣味でコードを書き、コードの質にこだわる事を最初から続けている。言語学と語学にも興味を持つ。}}

\newcommand{\jobexpEbay}{
	\jobexperience{\mlt{Software Engineer}{システムエンジニア}}
	{\mlt{Global Shipping Program}{国際出荷管理部} \& \mlt{New Product Development}{新商品開発部}}
	{eBay}
	{\mlt{September 2015}{2015年9月}}
	{\mlt{September 2019}{2019年9月}}
	{\begin{itemize}[noitemsep]
		\item \mlt{Developed from scratch, wrote tests for, and maintained sole ownership of the Scala distributed scheduler that powered proactive messaging for eBay ShopBot and eBay's Chinese iPhone app}
		{eBayのチャットボットと中国版iPhoneアプリの自発的なメッセージを送信する分散スケジューラとその自動テストをScalaでゼロから開発し、一人で保守}
		\item \mlt{Implemented a bootstrapping NLP process to extract informative noun/adjective pairs from reviews, using Python and Spacy (based on “RevMiner" paper)}
		{PythonでSpacyを利用し、レビューから有用な名詞と形容詞の組み合わせを抽出するブートストラップ自然言語処理を実装 (“RevMiner”という論文に基づき)}
		\item \mlt{Rearchitected the pipeline used to import shipping charge data from shipping partners, increasing the accuracy of shipping prices on eBay's search results page by up to 40\%}
		{提携先の運送会社からデータをインポートするパイプラインを再構築し、eBayの検索結果ページに載る配送料の精度を最大で40%改善}
		\item \mlt{Drove an effort to automate excessive manual operations work, saving the Global Shipping Program team tens of hours of work per week }
		{インフラやデータ整理などの手作業を自動化する取り組みを率い、国際出荷管理部チームの一週間当たりの手作業時間を数十時間減少させる事に成功}
		\item \mlt{Maintained and developed new features in Java/Spring services at scale (several million requests per hour), wrote tests using JUnit and Mockito}
		{JavaとSpringで大規模(毎時数百万のリクエスト )のウェブサービスにおける保守と新機能の開発をし、JUnitとMockitoを利用しその自動てテストも開発}
	\end{itemize}}
}

\newcommand{\jobexpGG}{
	\jobexperience{\mlt{Software Engineer}{システムエンジニア}}
	{\mlt{Global Growth}{グローバル成長}}
	{eBay}
	{\mlt{December 2018}{2018年12月}}
	{\mlt{September 2019}{2019年12月}}
	{\begin{itemize}[noitemsep]
		\item \mlt{Helped architect and develop features for the distributed scheduler that powers all of eBay's email campaigns}{eBayの全メールキャンペーンを送信する分散スケジューラの再構築に参画し、新機能を開発}
	\end{itemize}}
}


\newcommand{\jobexpNPD}{
	\jobexperience{\mlt{Software Engineer}{システムエンジニア}}
	{\mlt{New Product Development}{新商品開発部}}
	{eBay}
	{\mlt{November 2016}{2016年11月}}
	{\mlt{December 2018}{2018年12月}}
	{\begin{itemize}[noitemsep]
		\item \mlt{Developed from scratch, wrote tests for, and maintained sole ownership of the Scala distributed scheduler that powered proactive messaging for eBay ShopBot and eBay's Chinese iPhone app}
		{eBayのチャットボットと中国版iPhoneアプリの自発的なメッセージを送信する分散スケジューラとその自動テストをScalaでゼロから開発し、一人で保守}
		\item \mlt{Implemented a bootstrapping NLP process to extract informative noun/adjective pairs from reviews, using Python and Spacy (based on “RevMiner" paper)}
		{PythonでSpacyを利用し、レビューから有用な名詞と形容詞の組み合わせを抽出するブートストラップ自然言語処理を実装 (“RevMiner”という論文に基づき)}
		\item \mlt{Developed a Ruby on Rails web service that allowed non-technical users to customize and schedule proactive messages from ShopBot to users }
		{Ruby on Railsを利用し、非技術系社員が顧客へのメッセージをカスタマイズしたりスケジュール送信できるウェブアプリケーションをゼロから開発}
	\end{itemize}}
}

\newcommand{\jobexpGSP}{
	\jobexperience{\mlt{Software Engineer}{システムエンジニア}}
	{\mlt{Global Shipping Program}{国際出荷管理部}}
	{eBay}
	{\mlt{September 2015}{2015年9月}}
	{\mlt{November 2016}{2016年11月}}
	{\begin{itemize}[noitemsep]
		\item \mlt{Maintained and developed new features in Java/Spring services at scale (several million requests per hour)}
		{JavaとSpringで大規模(毎時数百万のリクエスト )のウェブサービスにおける保守と新機能の開発}
		\item \mlt{Produced tests for both new features and old untested features using JUnit and Mockito}
		{JUnitとMockitoを利用し、新機能とテストのない既存機能の自動テストを開発}
		\item \mlt{Drove an effort to automate excessive manual operations work, saving the team tens of hours of work per week }
		{インフラやデータ整理などの手作業を自動化する取り組みを率い、チームの一週間当たりの手作業時間を数十時間減少させる事に成功}
		\item \mlt{Rearchitected the pipeline used to import shipping charge data from shipping partners, increasing the accuracy of shipping prices on eBay's search results page by up to 40\%}
		{提携先の運送会社からデータをインポートするパイプラインを再構築し、eBayの検索結果ページに載る配送料の精度を最大で40%改善}
	\end{itemize}}
}

\newcommand{\jobexpGDIIntern}{
	\jobexperience{\mlt{Intern}{インターン}}
	{\mlt{Global Data Infrastructure}{国際データインフラ部}}
	{eBay}
	{\mlt{June 2013}{2013年6月}}
	{\mlt{September 2014}{2014年9月}}
	{\begin{itemize}[noitemsep]
		\item \mlt{Took ownership of and greatly expanded the functionality of a Ruby on Rails application serving eBay's data science teams, automating common support requests}
		{データサイエンス部からよく来るサポートリクエストを自動化するRuby on Railsウェブアプリケーションに取り組み、多数の新機能を実装}
		\item \mlt{Wrote or improved Ruby scripts to help with common ops tasks}
		{よくあるシステム運用の作業に役立つスクリプトを書き、または改善}
		\item \mlt{Performed troubleshooting and ops work in a Linux environment}
		{Linuxでシステム運用・保守}
	\end{itemize}}
}

\newcommand{\projectParasite}{
	\project{Parasite}{
	\mlt{A mod for StarCraft 2 which at peak averaged 2 million plays per week.  Written in a combination of a native GUI language and Galaxy, a C-like language. I was responsible for the architecture, design, and vast majority of the code written, eventually onboarding another developer and two community managers.}
	{一週間当たりにプレイされた回数が2百万回に達したStarCraft 2のMODである。専用のGUIプログラミング言語とGalaxyというC言語系プログラミング言語で開発を行った。一人で設計とデザインを考え出し、コードの大部分も一人で書いた。後でもう一人の開発者と二人のコミュニティ管理者でチームを結成した。}}
}

\newcommand{\projectPytext}{
	\project{Pytext}{
	\mlt{A desktop application written in Python which wraps communication with an email server and use of email to SMS gateways to allow users to participate in SMS conversations without the need for a phone number. }
	{ユーザーの入力したメッセージをメールとしてSMSゲートウェイに送信する事により、電話番号なしでSMS会話に参加する事を可能にするデスクトップアプリケーションである。一人でPythonで開発した。}}
}

\newcommand{\awardParasite}{
	\award{Parasite}
	{\mlt{Personal Project}{個人開発}}
	{\mlt{2009}{2009年}}
	{\mlt{A mod for StarCraft 2 which at peak averaged 2 million plays per week.  Written in a combination of a native GUI language and Galaxy, a C-like language. I was responsible for the architecture, design, and vast majority of the code written, eventually onboarding another developer and two community managers.}{一週間当たりにプレイされた回数が2百万回に達したStarCraft 2のMODである。専用のGUIプログラミング言語とGalaxyというC言語系プログラミング言語で開発を行った。一人で設計とデザインを考え出し、コードの大部分も一人で書いた。後でもう一人の開発者と二人のコミュニティ管理者でチームを結成した。}}
}

\newcommand{\awardAmplify}{
	\award{\mlt{eBay Amplify Presenter}{eBay Amplifyの登壇者に選出}}
	{\mlt{Paper Submission}{論文投稿}}
	{\mlt{2018/05/7}{2018年5月7日}}
	{\mlt{Chosen as a presenter for eBay's internal research conference (25\% acceptance rate). NLP research I had conducted as a side project won out over the submissions of full-time researchers.}{eBayの社内学会の登壇者に選ばれた (25%のアクセプト率)。片手間で進めた自然言語処理の研究が本業の研究者の論文と争い、勝ち抜いた。}}
}


\newcommand{\awardGoldenTicket}{
		\award{\mlt{eBay Innovation Expo Presenter}{eBayイノベーション博覧会の発表者に選出}}
		{eBay Hack Week}
		{\mlt{2016/06/24}{2016年06月24日}}
		{\mlt{One of two teams chosen from among 14 participating in Hack Week to present our work at eBay's Innovation Expo. We developed a Node.js service which selected n-grams from web pages to turn into eBay affiliate links using a heuristic based on relevance to page content.}
		{ウェブページのnグラムをページ内容への関連性に基づくヒューリスティックで選択し、アフィリエイトリンクに変換するNode.jsサービスを開発し、Hack Weekに参加した14チーム中から博覧会で作品を発表する2チームの1つに選ばれた。}}
}

\newcommand{\awardEbayApi}{
		\award{\mlt{Best eBay API Integration}{eBay API利用部門で優勝}}
		{2013 CodeDay Seattle}
		{\mlt{2013/05/26}{2013年05月26日}}
		{\mlt{Placed first in category among 7 teams by building a Python/Flask web application which signed into eBay and purchased random items from eBay on behalf of the user. }
		{ユーザーとしてeBayにログインし自動的にランダムな商品を購入するウェブアプリケーションをPythonで開発し、部門で7つのチームの中から優勝した。}}
}


\newcommand{\educationUW}{
	\education{\mlt{University of Washington}{ワシントン大学}}{\mlt{2019/09 – 2020/9}{2019年9月 – 2020年9月}}{\mlt{M.S.}{理学修士}}
	{\mlt{(Expected) masters degree in computational linguistics.}{(見込み)計算言語学の修士号}}
}
\newcommand{\educationEvergreen}{
	\education{\mlt{The Evergreen State College}{エバーグリーン州立大学}}{\mlt{2012/09 – 2015/4}{2012年9月 – 2015年4月}}{\mlt{B.S.}{理学士}}
	{\mlt{Graduated early with a major in computer science and a minor in Japanese.}{コンピュータ科学の専攻と日本語の副専攻で早期卒業}}
}

\newcommand{\educationHyogo}{
	\education{\mlt{University of Hyogo}{兵庫県立大学}}{\mlt{2014/4 – 2015/4}{2014年4月 – 2015年4月}}{}
	{\mlt{Studied Japanese language in a yearlong exchange program.}{日本語の勉強を中心とした留学プログラム}}
}

\newcommand{\skillenglish}{\mlt{English - Native}{英語 - 母国語}}
\newcommand{\skilljapanese}{\mlt{Japanese - Fluent (JLPT N1)}{日本語 - 日本語能力試験N1}}